\section{Introduzione}
In questo progetto si andrà ad analizzare gli abusi sessuali sui minori. Un abuso sessuale é tutto ciò che riguarda un comportamento sessuale non consensuale; ovvero non concordato da tutti gli attori in questione. Questo atteggiamento ha diverse forme con i rispettivi termini tra loro differenti ma la focalizzazione si terrà sugli abusi sessuali sui minori. 
Con il seguente termine si intente ogni tipo di contatto sessuale non concordato con un minore ossia che non ha ancora raggiunto la maggiore età. Una chiara definizione di “abuso sessuale” è data da Henry Kempe, pediatra e virogolo, esso fu il primo nelle comunità mediche per identificare e riconoscere l’abuso sui minori. Secondo Kempe (1962), “il coinvolgimento di bambini e adolescenti, soggetti quindi immaturi e dipendenti, in attività sessuali che essi non comprendono ancora completamente, alle quali non sono in grado di acconsentire con totale consapevolezza o che sono tali da violare tabù vigenti nella società circa i ruoli familiari”.
L’abuso sessuale sui minori, pur essendo esistito in ogni epoca storica, è stato riconosciuto solo recentemente come grave fenomeno, che deve coinvolgere l’attenzione di diverse figure professionali tra i quali: avvocati, forze dell’ordine, medici, psicologi e operanti sociali. Saranno dunque degli attori in caso di presenza di abuso sessuale sui minori. Questo tipo di violenza è diffusa soprattutto nell’ambito della famiglia, un ambiente in cui si dovrebbe tutelare e sostenere affettivamente e materialmente il bambino. L’abuso sessuale ha riscontrato un grande ritardo nel riconoscerlo come grave fenomeno da bloccare e come reato da perseguire penalmente, dal momento che quest’atto produce nella vittima gravi conseguenze psicologiche e fisiche. 
Lo sforzo che risulta più difficile collegato a questo fenomeno, è quello rivolto ad una prevenzione primaria più efficace, ma soprattutto volto ad una sensibilizzazione per riuscire a cambiare l’attegiamento di controllo e di abuso che gli adulti tendono a riproporre nei confronti dei minori.
La domanda che ci siamo posti è la seguente: “la prevenzione e la sensibilizzazione in Ticino sono sufficienti per impedire gli abusi sessuali?”.
Domanda che ha luogo in Ticino in quanto ci sarà l’analisi delle diverse associazioni e tipologie di prevenzioni ticinesi.
