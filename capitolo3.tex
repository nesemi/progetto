\section{Tipologie di abuso sessuale sui minori}
L’abuso sessuale sui minori riguarda un comportamento sessuale che prevede una discordanza sull’azione da compiere da parte di uno degli attori che è chiamata vittima poiché è sottoposta a qualcosa che va oltre al suo volere e potere. Ci sono diverse tipologie che appartengono a questo fenomeno, con attori differenti (Colesanti & Lunardi, 1995, pp. 70-80).
\subsection{Abuso sessuale intrafamiliare}
La tipologia più diffusa è l’abuso intrafamiliare, la quale rappresenta un abuso attuato dai membri del nucleo familiare, quali genitori (compresi quelli adottivi e affidatari), patrigni, matrigne, fratelli o da membri della famiglia allargata quale nonni, zii, cugini o amici stretti di famiglia. L’abuso intrafamiliare è la tipologia più frequente tra tutte, inoltre al suo interno, vi sono riconosciute molteplici modalità di realizzazione, tanto da poterli distinguere in ulteriori tre sottogruppi: gli abusi sessuali manifesti, gli abusi sessuali mascherati e gli pseudo-abusi. Per quanto riguarda il sottogruppo degli abusi sessuali manifesti, vi appartengono tutte le attività sessuali che prevedono un contatto fisico tra la vittima e l’aggressore, inoltre, sono inclusi tutti i comportamenti sessuali che; coinvolgono terze persone (ad esempio i fratelli), comportamenti sessuali fatti compiere o subire dal minore mediante obbligazione. Il secondo sottogruppo, relativo agli abusi sessuali mascherati, è caratterizzato da attività genitali insolite, quali ad esempio i frequenti lavaggi di genitali, le ispezioni ripetute, l’applicazione di creme e fare interventi medici per emergenti problemi urinari e genitali. Attraverso queste attività, l’abusante giustifica e maschera i vari toccamenti e sfregamenti attraverso i quali si procura un eccitamento sessuale fisico o fantastico. Il terzo sottogruppo, relativo agli pseudo-abusi, riguarda gli abusi dichiarati quando in realtà non sono mai accaduti. Una dichiarazione che non risponde alla realtà dei fatti accaduti si ascolta frequentemente da madri o nonne contro i padri nel corso delle separazioni e divorzi. Il bambino in queste situazioni tende ad adeguarsi o viene obbligato ad affermare le accuse inventate da uno dei genitori (Colesanti & Lunardi, 1995, pp. 70-80).
\subsection{Altre tipologie di abuso sessuale}
Un altro tipo di abuso sessuale, seguito da quell’intrafamiliare, è quella extrafamiliare. Questa tipologia è la seconda più frequente e intende l’abuso attuato da persone conosciute dal minore, quali vicini di casa e conoscenti. Troviamo anche l’abuso sessuale assistito la quale ha atto quando il minore assiste a un abuso sessuale che un genitore esegue su un fratello o una sorella, oppure quando egli viene fatto assistere alle attività sessuali dei genitori. Altre tipologie di abuso sessuale possono essere quelle istituzionali; che comprende un abuso attuato da persone alle quali i minori sono affidati per ragioni di cura, custodia, educazione, gestione del tempo libero, all’interno di diverse istituzioni e organizzazioni (insegnanti, medici, assistenti di comunità, allenatori,..). Abuso sessuale di strada, ossia un abuso attuato da parte di persone sconosciute. Abuso sessuale ai fini di lucro; un abuso commesso da parte di singoli o gruppi criminali organizzati, quali le organizzazioni per la produzione di materiale pornografico, per lo sfruttamento della prostituzione o agenzie per il turismo sessuale. Infine un abuso sessuale da parte di gruppi organizzati ossia abuso da gruppi esterni al nucleo familiare come sette e gruppi di pedofili.
Le diverse tipologie si differenziano dagli attori in questione; in particolare c’è la differenza della persona abusante (Colesanti & Lunardi, 1995, pp. 70-80).
