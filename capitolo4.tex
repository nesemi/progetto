\section{Le conseguenze fisiche e psichiche sulla vittima}
Il bambino in età infantale non è in grado di comprendere l’esperienza subita, pertanto non riesce a comunicare verbalmente quello che è accaduto, anche perché non possiede un vocabolario inerente al comportamento sessuale adulto; ciò lascia il bambino in uno stato di confusione e di disorientamento riguardo alle esperienze subite. In particolare, la sintomatologia del bambino abusato sessualmente in età infantile è caratterizzata da: sintomi fisici, comportamentali, affettivi e cognitivi.
Tra i sintomi fisici troviamo tutto ciò che riguarda un disturbo fisico visibile e non; come le infezioni veneree e urinarie, i lividi sul corpo ed emorragie. Tra i sintomi comportamentali i disturbi più evidenti sono: disturbi del sonno (incubi o terrori notturni), disturbi dell’alimentazione (anoressia o bulimia), disturbi della condotta (in particolare aumento dei comportamenti autodistruttivi, aggressivi e oppositivi), fughe da casa. I sintomi affettivi sono quelli più frequenti e sono: ansia, timidezza estrema, sensi di colpa, paura del fallimento, ostilità, aggressività con il gruppo dei pari, basso livello di autostima, difficoltà di apprendimento, rabbia intensa, situazioni di panico totale, sentimento di vergogna, paura di vedere determinate persone e di andare in certi posti, depressione, isolamento, atti delinquenziali, abuso di sostanze, tentativi di suicidio. Infine ci sono i sintomi cognitivi che comprendono i problemi scolastici e disciplinari, disturbi dell’apprendimento e nell’andamento scolastico (Violenza sessuale e disturbi mentali: le conseguenze del trauma, 2008).
