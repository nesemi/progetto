\section{Gli indicatori e i segnali dell'abuso infantile}
Gli indicatori e segnali trasmessi dalla vittima dell’abuso sessuale formano una guida per accertare l’esistenza dell’abuso, che in tal caso è necessario un intervento e un trattamento psicologico. 
Gli indicatori, considerati a proposito della vittima, sono di tre tipi: cognitivo, fisico e comportamentale ed emotivo (De Leo & Petruccelli, 1999, pp. 82-87). 
\subsection{Indicatori cognitivi} 
Tra gli indicatori cognitivi possono rientrare le conoscenze sessuali inadeguate per l’età, il modo di rilevazione da parte del bambino della conferma dell’abuso sessuale, i dettagli dell’abuso sessuale con tutti i particolari. Questi indicatori sono presenti anche nel racconto dell’accaduto della vittima; se viene individuata una certa confusione nel ricordo dei fatti e sovrapposizione dei tempi. Le aree da indagare, ai fini della scoperta d’indicatori di tipo cognitivo, riguardano pertanto il livello di coerenza delle dichiarazioni, l’elaborazione fantastica e la chiarezza semantica (De Leo & Petruccelli, 1999, pp. 82-87).  
\subsection{Indicatori fisici}
Gli indicatori fisici di abuso sessuale vanno da quelli più evidenti a quelli più nascosti e incerti. Nei segnali visivi ne fa parte la deflorazione, la rottura del frenulo, i lividi in zona perineale, malattie sessualmente trasmesse o malattie veneree, forti lesioni all’apparato genitale o anale (come pruriti, dolore, emorragie, infezioni, graffi,..) Gli altri segnali la quale non sono ben visibili sul corpo, devono considerarsi più incerti in quanto possono esserci per molteplici cause, tra cui possono essersi generati per cause naturali. Tra questi ultimi vanno inclusi anche le neo vascolarizzazioni a livello del derma e delle grandi labbra ovvero le irritazioni del glande o del prepuzio, nonché gli arrossamenti e le infiammazioni aspecifiche localizzate (De Leo & Petruccelli, 1999, pp. 82-87).
\subsection{Indicatori comportamentali ed emotivi}
A far parte degli indicatori comportamentali ed emotivi troviamo vari sentimenti: di paura, di depressione, disturbi del sonno e dell’alimentazione. Inoltre la vittima tenderà ad avere un comportamento molto vigilante e attento rispetto a com’era prima dell’abuso. Ciò indica la paura di un’eventuale ripetizione del trauma, la mancanza d’interesse su diverse attività, la rilevante alterazione della personalità con possibili sintomi psiconeurotici (tra cui isteria, fobie, ipocondria). Tutti gli effetti comportamentali causati dall’abuso sessuale possono essere riscontrati non solo per un breve periodo ma anche a lungo termine. 
A causa dei sensi di colpa, delle accuse o minacce che ricevono i bambini sessualmente abusate, essi possono mettere in atto comportamenti autodistruttivi, in alcuni casi arrivare addirittura al suicidio. Bisogna tener presente che non tutti i bambini, il quale, vittime di abuso sessuale, presenta segni di lesione nell’area intima o altri segni visibili, tuttavia tanti casi di abuso restano ignoti alle autorità, finché la vittima stessa s’incoraggia e denuncia l’accaduto oppure un esperto scopre ciò che è avvenuto. La possibilità di individuare atti di abuso sessuale su un minore, pertanto dipenderà dall’attenzione che i diversi operatori pongono su alcuni indicatori sia fisici sia comportamentali. Talvolta alcuni segni costituiscono una vera e propria prova tangibile di un abuso sessuale; mentre altri segni, soprattutto quelli comportamentali, si possono presentare per vari motivi risultando perciò indefinibili. Un abuso sessuale può aumentare i motivi di sospetto quando c’è la manifestazione di più sintomi messi insieme (De Leo & Petruccelli, 1999, pp. 82-87). 
