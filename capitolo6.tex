\section{Diagnosi psicologica}
Per determinare se effettivamente è avvenuto un abuso sessuale infantile c’è bisogno della diagnosi clinica e se viene affermato l’atto si provvederà alla protezione e di un trattamento psicologico per la vittima se essa la necessita. Delle linee guida per la diagnosi psicologica è stata strutturata dall’American Academy of Child and Adolescent Psychiatryche, esse possono aiutare il clinico nella diagnosi per un eventuale trattamento. Per compiere la valutazione c’è bisogno di un professionista con molta esperienza e competenze speciali riguardo l’abuso sessuale. Il clinico in questione dovrebbe avere una specifica conoscenza sulle teorie dello sviluppo del bambino, sulle caratteristiche delle dinamiche familiari relative all’abuso e sugli effetti dell’abuso sessuale infantile. Inoltre deve essere allenato nella valutazione diagnostica dei bambini e deve sentirsi in grado, in caso di bisogno, di testimoniare in tribunale. È quindi necessario che il clinico abbia una formazione professionale specifica (De Leo & Petruccelli, 1999, pp. 87-90).
\subsection{Caratteristiche del luogo dell’intervista}
Per l’incontro della vittima e del clinico ci sono vari fattori da considerare come il luogo. Si richiede che il luogo dell’intervista sia rilassante, neutrale e che garantisce la privacy del bambino, evitando qualsiasi tipo d’interruzione (telefonica, persone che entrano della stanza). Inoltre, per eseguire al meglio l’intervista, è necessario ottenere il resoconto della storia dai genitori o da coloro che si prendono cura del minore e riuscire ad avere informazioni sulla storia medica evolutiva e cognitiva della vittima. Inoltre, il clinico deve focalizzare l’attenzione su eventuali abusi precedenti e sulle storie di un eventuale abuso da parte di un o entrambi i genitori. Durante l’intervista può esserci la presenza della video registrazione. La video registrazione è molto consigliata siccome grazie ad essa si può preservare le affermazioni iniziale del bambino, evitare alla vittima più interviste (poiché è sufficiente mostrare il video), evitare che il bambino debba testimoniare, infine la video registrazione è un efficace strumento didattico. L’utilizzo della telecamera però può presentare anche delle difficoltà e il clinico necessariamente deve prendere in considerazione gli eventuali rischi. La registrazione tramite telecamera può causare un’intimidazione al bambino e quindi far si che esso non si esprima. Inoltre un altro problema riguardo la video registrazione può essere che viene mostrata fuori dal contesto e finisce nelle mani di coloro che non hanno l’obbligo del segreto professionale. In ogni caso il bambino deve essere informato sullo scopo della video registrazione (De Leo & Petruccelli, 1999, pp. 87-90).
\subsection{Test del disegno}
Uno dei metodi molto utilizzati per la diagnosi di un abuso sessuale infantile è il test del disegno, il quale può risultare molto utile. Questo tipo di test è usato per le vittime di età infantile che non hanno ancora la capacità di esprimersi al meglio in modo verbale. Si utilizza soprattutto il disegno della famiglia e della figura umana e l’utilità dei disegni sta nelle informazioni che essi esprimono. Alcuni di essi possono indicare che c’è stato un abuso come ad esempio; il disegno di genitali molto evidenti, il disegno della vittima a letto con una persona adulta oppure far mancare organi di senso nel proprio disegno ad esempio gli occhi che indica il non voler vedere. Dunque il test del disegno è un linguaggio simbolico e diventa una comunicazione alternativa a quella verbale e deve essere interpretata da uno psicologo esperto in materia così da riuscire ad individuare i significati presenti (De Leo & Petruccelli, 1999, pp. 87-90).
\subsection{Test con bambole anatomiche}
Alcuni clinici, per effettuare il test di diagnosi, utilizzano le bambole anatomiche. Esse non sono indispensabili ma possono essere utili per permettere al bambino di mostrare cosa è successo e in che modo, qualora esso non sia in grado di disegnare o di raccontare. Bisogna fare attenzione a non usarle in maniera da istruire o condurre il bambino a nuove conoscenze (De Leo & Petruccelli, 1999, pp. 87-90).
\subsection{Comportamento del clinico verso la vittima}
Il clinico, per la fase d’intervista, deve comportarsi in un modo specifico. Egli deve mantenere una neutralità emotiva e affrontare il caso con apertura mentale, adottando un atteggiamento non giudicante. È importante sapere che il clinico non deve porre domande che indirizzino la risposta e non deve obbligare o forzare la risposta; il bambino deve raccontare la storia con le sue stesse parole. Il clinico deve focalizzarsi sulle descrizioni dettagliate degli eventi che la vittima pone e determinarli nel tempo, luogo e frequenza. È utile tornare più volte sugli stessi eventi poiché potrebbero emergere ulteriori informazioni (De Leo & Petruccelli, 1999, pp. 87-90). 
\subsection{Fattori di credibilità della vittima}
Ciò che aumenta la credibilità di un minore sono i racconti dettagliati, la spontaneità, l’inclusione di dettagli per lui turbanti, la consistenza del racconto a lungo termine (alcuni dettagli e descrizioni potrebbero cambiare ma il racconto del bambino sugli eventi sostanzialmente rimane lo stesso), i cambiamenti comportamentali dopo l’abuso sessuale (De Leo & Petruccelli, 1999, pp. 87-90).
