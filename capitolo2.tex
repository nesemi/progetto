\section{Cenni storici}
L’abuso sessuale sui minori affonda le sue radici in un tempo molto antico, è quindi un fattore presente da sempre fino ai giorni d’oggi. Tuttavia è sbagliato il pensiero che questo fenomeno è tipico della società contemporanea. Ogni era, infatti ha dettato le proprie leggi sul comportamento sessuale contro i più deboli e i meno benestanti. Per quanto riguarda la prevenzione e la protezione nei confronti dei minori possiamo affermanre che in tutte le società si riscontrano dei sistemi appositi. La prima società nazionale per la prevenzione sugli abusi dei minori fu fondata a Londra nel 1884;  decisero di assumere delle équipe di ispettori che controllavano le condizioni dei minori nei quartieri a rischio. Successivamente la prevenzione utilizzata venne diffusa in tutte le società. 
Gli studi e le ricerche sul tema dell’abuso e del maltrattamento all’infanzia hanno vissuto varie fasi: come visto inizialmente l’interesse era focalizzato sull’abuso fisico ossia sul maltrattamento, tipologia più facilmente riconoscibile poiché lascia segni più evidenti, dagli anni ’80 l’attenzione ha iniziato a spostarsi sull’abuso sessuale a danno di minori, mentre solo più recentemente l’abuso psicologico e la trascurezza sono divenuti oggetto di studio. Infatti solo in un secondo momento nascono scienze come la pedagogia, la psicologia e la sociologia che cominciarono ad interessarsi al problema dell’infanzia e ai bisogni dei minori. Al bambino vennero riconosciuti bisogni affettivi e psicologici e venne sottolineato che i diritti del minore dovevano essere tutelati non solo dai genitori, ma anche da tutta la società. In quest’ottica nel 1925 venne approvata a Ginevra la “dichiarazione dei Diritti del Fanciullo” in cui si affermava che il fanciullo doveva essere posto in condizione di svilupparsi in maniera normale, sia sul piano fisico che su spirituale, dal momento che egli aveva diritto di essere nutrito, curato, soccorso e protetto da ogni forma di sfruttamento e maltrattamento. Negli anni ‘40 e ‘50 la classe medica cominciò a prendere coscienza del problema dell’abuso infantile. Proprio in quegli anni comparivano nella letteratura scientifica le prime descrizioni di bambini picchiati e maltrattati. In seguito a queste scoperte nel 1959, viene proclamata dall’Assemblea dell’Onu la “Carta dei diritti del Fanciullo”, nella quale viene ribadito il diritto di nascita, e di cure adeguate alla madre e al bambino nel periodo pre e post-natale; la protezione dalle discriminazioni razziali e religiose e il diritto di poter vivere in un clima di comprensione e tolleranza. Nel gennaio 1986 il Parlamento Europeo ha approvato le stesse raccomandazioni del precedente documento, con una particolare attenzione al problema dell’abuso infantile e alla necessità di protezione del minore. Infine il Consiglio d’Europa, nel gennaio 1990, esprime la necessità di misure specifiche di informazione, di individuazione delle violenze, di aiuto e terapia a tutta la famiglia e di coordinamento tra i servizi.
In Svizzera sino gli anni ottanta, le autorità della confederazione, con la motivazione ufficiale di proteggere i minori, tolsero ai genitori che si trovavano in situazioni di difficoltà migliaia di bambini, affidandoli a famiglie spesso operaie o contadine che avevano bisogno di manodopera a basso costo. Questi bambini; denomitati schiavi-bambini per via del loro impiego, molto spesso furono oggetto di maltrattamenti ed a loro non venne assicurata l’educazione (Abuso sessuale sui bambini: cenni storici, mitologie e criteri di definizione, 1997).
