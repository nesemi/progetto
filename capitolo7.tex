\section{La prevenzione}
\subsection{Lavoro di prevenzione all’interno della scuola}
Il lavoro di prevenzione sugli abusi sessuale è molto importante poiché informa sui rischi ed aiuta a ridurli. La scuola rimane il luogo più privilegiato per attuare un lavoro di prevenzione del fenomeno dell’abuso, poiché, dopo la famiglia, è l’ambiente dove i ragazzi passano la maggior parte del loro tempo. A scuola imparano a convivere e a condividere, dunque imparano a crescere ed è quindi essenziale informare gli scolari su eventuali rischi di abuso sessuale al fine di istruirli e far sì che non hanno alcun pericolo. Il lavoro di prevenzione a scuola sugli abusi sessuali risulta avere una certa debolezza poiché non è ancora efficace al 100%. All’interno della scuola il problema che troviamo è di natura psicologica. Succede che il fenomeno di abuso sessuale risulta un argomento forte e sconvolgente per tutti e dunque è difficile affrontare il tema. Gli attori presenti a scuola, ossia gli operatori scolastici, si dovrebbero prendere cura della prevenzione ma ciò gli risulta difficile per via del fragile tema e tendono quindi a rimuovere l’argomento e tuttavia eliminare la prevenzione sul fenomeno che diventa sempre più grave. 
Il vero compito di un insegnante, oltre alle normali competenze didattiche, è di possedere le capacità empatiche così da oltrepassare l’aspetto di disagio e imbarazzo del fenomeno traumatico e di parlare delle situazioni a rischio. I docenti accompagnano la crescita dei ragazzi e perciò entrano facilmente in contatto con le loro esperienze traumatiche tra cui possibili abusi sessuali. In tali circostanze è fondamentale il rapporto instaurato tra insegnante e alunno perché l’alunno riuscirà ad esprimersi per via della relazione positiva col docente. Tuttavia il modo di esprimersi può essere anche non verbale e quindi mettere in atto segnali evidenti che permette al docente di accorgersi del cambiamento. Per esempio la tendenza all’isolamento, frequenti scoppi di piano o di aggressività fisica, il rifiuto del contatto fisico, un evidente calo nel rendimento scolastico. Il docente fa le sue considerazioni e prova a interpretare il caso e a dipendenza delle motivazioni che si da decide se intervenire oppure lasciare il caso in sospeso. Più frequentemente i professori interpretano questi atteggiamenti come non motivazione verso la materia questo perché nella pratica scolastica esiste una realtà caratterizzata da scarsa conoscenza e preparazione riguardo al problema e una profonda incertezza nel modo di gestione del caso e quindi i professori hanno sempre meno l’idea che dietro a quegli atteggiamenti dell’alunno ci possa essere un abuso sessuale. Inoltre una spiegazione per i docenti riguardo ai protocolli d’intervento in caso di scoperta di un abuso sessuale è assente. Di conseguenza se si vuole un’efficace lavoro di prevenzione dell’abuso sessuale all’interno della scuola è possibile solo attraverso un’efficace e continuo lavoro di formazione degli insegnanti.
Pertanto si è cercato di inserire la formazione del lavoro di prevenzione agli operatori scolastici e di fargli acquisire competenze utili al rilevamento dell’abuso non ancora avvenuto (Violenza sessuale e disturbi mentali: le conseguenze del trauma, 2008).
\subsection{Il lavoro di prevenzione all'esterno della scuola}    
Si presenta un’altra debolezza del lavoro di prevenzione di natura esterna alla scuola: manca la comunicazione e il collegamento tra scuola e figure istituzionali, le quali si mettono a disposizione ad affrontare il problema, inoltre manca una politica scolastica che obbliga ad attuare una vera e propria formazione sul tema.
Il rapporto tra la scuola e le istituzioni esterne quindi risulta essere assente e questo porta a delle complicazioni. La denuncia o la segnalazione parte dalla scuola e arrivata alle istituzioni si studia il caso; esse tendono a prendere il caso con leggerezza e quindi da diminuire la gravità; di conseguenza si metteranno in atto metodi poco efficaci se non del tutto controproducenti. Manca quindi una reale condivisione delle procedure e l’attuazione di un piano unitario d’intervento. Inoltre manca anche il sostegno per la vittima d’abuso per ciò che concerne la denuncia, il trattamento e l’accertamento legale (Violenza sessuale e disturbi mentali: le conseguenze del trauma, 2008).  
